\textcolor{blue}{
\textbf{Deliverables:}
\begin{itemize}
\item requirements $\xrightarrow{\rm status}$ draft
\item concept design (if necessary baseline and backups) including 3D model
\item main questions to answer in this subsection
    \begin{itemize}
    \item number of coils
    \item coil size
    \item power consumption
    \item envelope for inductance
    \item active component?
    \item minimum number of magnetometers
    \item achievable sampling frequency, ...
    \end{itemize}
\item calculations proving that the design concept fulfills the requirements 
\item interface definitions (detailed)
\item design and manufacturing plan
\item schedule
\end{itemize}
}

\subsubsection{Requirements}

\paragraph*{RS 3.-34}	The AMC shall reduce the external magnetic field to a level comparable to Earth magnetic field, less than 50 muT 
\newline Rationale: 	
%The passive shielding of the MSR can only handle a field of a certain magnitude, so the external field must be compensated.
We do not expect the outer layer of the MSR to saturate (refer to A Sher's calulations to be inserted into this document), however, the manufacturer of the MSR will not certify it's performance for any external field value larger than Earth field
\newline Rationale: 	The 50 muT requirement is Earth’s field, but a smaller target field may be desirable
\paragraph*{RS 3.-35}	The AMC shall be constructed in such a way that it does not prevent access to the MSR. A 'door' as well as a roof lid might be required in the AMC coil cage.
\newline Rationale: 	It will certainly be necessary to enter the MSR throughout the experimental run, so the AMC cannot render this impossible.
\paragraph*{RS 3.-36}	The ambient field of the experimental area shall be mapped and monitored to a precision acceptable to specify the construction of the AMC; 
\newline Rationale: 	It is important to understand the ambient magnetic conditions due to other magnetic equipment prior to running the experiment. ie the maximum DC values and amplitude of AC changes need to be known such that the right power and bandwith/speed of power supplies as well as inductance of coils can be chosen/determined.
\paragraph*{RS 3.-37}	The ambient magnetic field control system shall be developed such that its cost falls within the CFI budget allocation.
\newline Rationale: 	 This is necessary for completion of the experiment.
\paragraph*{RS 3.-38}	Development of the ambient magnetic field control systems shall be completed in the timeframe given by the Level 1 schedule shown in Document-154393. Critically this specifies installation of all hardware by 2021.
\newline Rationale: 	This timeline is necessary to begin data taking in 2021 in order for the TUCAN EDM experiment to remain competitive.

\paragraph*{Requirement:} buck the cyclotron field to a level that fulfills IMEDCO operations conditions
for the MSR (below Earth field) and ensure the outermost layer will not saturate;
\paragraph*{Requirement:}
maybe stabilize the external field in subHz region if that can improve the MSR performance

\subsubsection{Other stuff}

\paragraph*{Open question:} Final conclusion of necessity of dynamic part depends on manufacturer specs, and might
only be accessible once the MSR is installed at TRIUMF and tested

\paragraph*{Baseline design:} A coil cage of three orthogonal ‘merrit coils’ should do the job, but it should be checked that
sufficient orders of magnetic field can be covered by this kind and number of degrees of
freedom
\paragraph*{Design Status:} Simulation and design note needed, maybe tad bit more
R\&D
\paragraph*{Interfaces:} Relation to thermal enclosure, inside or outside? Does their mechanical structure interfere?

\subsection{DAQ/AMC Interface Requirements}

Q1: Will your subsystem be providing a data stream (over some network connection) to the DAQ/controls?  Or will you be providing an analog signal(s) that you want digitized by some DAQ/controls module? 
If you are providing only a data stream, then you can skip question 2.

A1: The AMC will provide a data stream


Q3: Does your subsystem require a digitizer that is referenced to a central atomic clock?

A3: A syncronization at the level of $\sim1$\,s should be good enough for AMC, so I don't see a necessity for a reference to an atomic clock. TBC though.

Q4: Does your subsystem have controls or gas handling like equipment?

Q4.1 If yes does your equipment require PLC-type interlocks to ensure correct/reliable operation?  Ie is there equipment-protection, safety or operational reasons why controls needs to be done with a PLC-type system? Or is it sufficient to have a computer program doing the control?

A4: I currently don't see a reason for a PLC controlled system for the AMC. Might need to think about it again later when the design is a little further fleshed out.







